\documentclass[letter,12pt]{article}

%---------------------------------------------------------------
\usepackage{listings} % allow to include nicely formated listings
\usepackage{color}    % we can add color
\usepackage{graphics}
\usepackage{fullpage} % default article style is to greedy about margins
\usepackage{amsmath}
\usepackage{amsthm}
\usepackage{amssymb}
\usepackage[utf8]{inputenc}
\usepackage{amsfonts}
\usepackage{mathtools}
\usepackage{algorithm}
\usepackage{listings}
%---------------------------------------------------------------
\begin{document}
%---------------------------------------------------------------
\title{Robotics Final Project: Flying Through a Ring}
\author{Cristian Gonzalez, T00902102\\
Holger Rasmussen, A00922103\\
Barbara Sepic, A00922104,\\
Felix Stahlberg, A00922105
}
\maketitle

\section*{Problem Statement:}
A main benefit of flying robots over other forms is the enhanced mobility. Taking advantage of it requires sophisticated motion control, adjusted trace finding algorithms and methods capable of recognizing obstacles in an airspace. We address all of the stated challenges in following scenario: Using quadrotor’s front-camera, we identify the position and orientation of a ring and fly through it following a suitable trace.  If time allows it, we can change the static environment into a dynamic environment by moving the ring around while the quadrotor is observing it or even use a set of rings instead of just one.


\section*{Approach:}
After the ring is placed in the space, the robot takes pictures from different positions and builds its model of the ring position and orientation that way. Using this information, a trace from starting point is calculated passing the midpoint of the ring in a certain direction ending up to some defined endpoint (for instance using a Bezier curve approach). Then the robot follows this trace using the Android API.

\section*{Division:}
\begin{itemize}
	\item \textbf{Quadrotor:} Barbara, Cristian, Felix, Holger
	\item \textbf{Project Infrastructure:} Felix, Holger
	\item \textbf{Project Mangagement:}	Cristian
	\item \textbf{Project Presentation:} Barbara
\end{itemize}

\end{document}